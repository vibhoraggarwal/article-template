% Cover letter using letter.cls
\documentclass{letter} % Uses 10pt
\usepackage{hyperref}
\usepackage{tikz}
\usepackage{graphics}
\usepackage{xspace}
\usepackage{svg}
\topmargin=-1.5in    % Make letterhead start 1 inch from top of page 
\textheight=10in  % text height can be bigger for a longer letter
\oddsidemargin=0pt % leftmargin is 1 inch
\textwidth=6.5in   % textwidth of 6.5in leaves 1 inch for right margin
\pagestyle{empty}

\date{ } % Uncomment for today %used in the letter format but not for an essay

\begin{document}
\longindentation=0pt                       % needed to get closing flush left
\let\raggedleft\raggedright                % needed to get date flush left
 
\begin{letter}


\begin{center}
{\large\bf Vibhor Aggarwal} 
\end{center}
\medskip\hrule height 1pt
\begin{center}
{Motivation letter } 
\end{center} \vspace{0.5 cm} % forces letterhead to top of page
I grew up feeling fascinated by things that improves the way we perform the job at hand. Continuously solving real life problems and playing with the mechanical advantages of the objects is what motivates and drives me forward.

I started making robots while I was a freshman at college. I was interested to learn about the art of machines. After serving a stint with the robots, I worked on the vehicle dynamics with college’s Motorsports team, which I eventually led, as the captain. It was then; I felt there is something that I cherish about dynamics of machines. We made two off-road vehicles in two successive years. I was involved with calculation of the Suspension and the steering parameters, their design and modeling on Matlab and doing its Finite Element Analysis on Ansys. During the summer of sophomore year, I interned with an organization designing robotic systems for warehouses’ automation, wherein my task included working on robot’s dynamics and designing a new suspension design, adhering to Design for Manufacturing and Assembly guidelines.
 
It was not until I worked on my undergraduate project that I decided to invest more time on the Robotic systems. Under the supervision of Prof. Sumit Basu from the Mechanical Engineering department at Indian Institute of Technology Kanpur I worked on the “Robotic Exoskeleton Arm”. It was aimed at bio-mimicing human muscles using latex, restricting it axially, using Polyethylene Terephthalate based flexo, and controlling it through voice commands sent using Bluetooth via an Android application and an Arduino. It could help patients suffering from Cerebral Palsy in the physiotherapy of their limbs, and can help strengthen old age mobility. For the project, I received two awards at the convocation ceremony viz “Ranjan Kumar Memorial award” for the best socially relevant project, and “The Proficiency medal” for the best undergraduate project work from the class of 2017 at IIT Kanpur. 

As a fresh graduate, I worked in the Engine Plant Maintenance department of Hero MotoCorp Ltd. I managed the Total Productive Maintenance for machinery equipment and quality related activities and completed operations pertaining to maintenance repair concerning resource planning and in-process inspection. 

Soon forth, I worked as a research fellow in Italy under Dr. Daniele Pucci at the Italian Institute of Technology, Genova. I worked on the humanoid robot, iCub, with major tasks to define and identify the discrete time transfer function between the voltage applied and the torque acting on each joint of iCub. Thereby, created a computational framework that is effective in torque control of the joints using the position, velocity and pre-measured torques.

Currently, I am attending a master's program in Automotive Engineering at RWTH Aachen, Germany. Alongside I am working as a Student research assistant with a PhD student on the sensor fusion for the navigation and path planning for an Unmanned Aerial Vehicle.

% identificaiton of fundamental research problems by understanding the current literature
% TODO: Mention the topic of interest in the following paragraph.

The research explorations I have undertaken, the platforms I have worked with, and the skills I have developed from the previous projects has helped me in gaining proper experience to work in the field of Robotics and Autonomous systems. Thus, making me an asset for the research group. Through a research position at your lab I can dwell deeper into the world of robotic systems. This would be essential for realizing my goal of becoming a problem solver in the field.

%I want to build the future for urban mobility for which I tend to work at the border line between Robotics and Automotive research.
%I believe that my academic background, industrial experience and my zeal for research will help me be successful as a graduate scholar 
%TODO: enter the name of the university
I am interested to work as a research fellow for study on sensor fusion and algorithm design for robotic systems at  Nanyang Technological University, Singapore. Working with highly talented and enthusiastic people involved in the field of robotics would help me grow both professionally and technically. Moreover, during the program I would get the opportunity to interact with one of the best students and faculty for multidisciplinary research. In future, I want to work on the robots that are not just capable of working in environment independent of the humans but will also able to collaborate with humans in complex task. For which, I believe, the program can serve me with the right knowledge, tools and facility.

%\pagebreak
%{\large\bf Vibhor Aggarwal}\hfill Statement of Purpose for position of a Visting student

%\medskip\hrule height 1pt

%I want to focus on the topic of Automated driving and robotics manipulation, and how I could bring the best out of the two. Further than that, I would be more interested in joining a research group,which works on the same topic of increasing the human robot interaction or improving the current urban mobility system and biomedical robotics research.



%I believe that with the oncoming of automation; automotive and robotic industries will not be different from each other. A future vehicle would be just like a programmable robot allowing us to go from A to B with or without our intervention.

% TODO 1: change the name of the university in the following paragraph
% TODO 2: change the name of the professor(if any)
% TODO 3: change the name of the research topic the professor is working in.

\end{letter}
\end{document}
